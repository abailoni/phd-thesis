% !TEX root = ./main.tex
\usepackage{wrapfig}
\usepackage{amsfonts}
\usepackage{amsmath,amssymb}
\usepackage{subcaption}
\usepackage{epsfig}
\usepackage{tikz}
\usepackage{bbm}
\usepackage{mathrsfs}
\captionsetup{compatibility=false}
% \usepackage[caption=false]{subfig}

\DeclareMathOperator*{\argmax}{arg\,max}
\DeclareMathOperator*{\argmin}{arg\,min}
\usepackage{mathrsfs}
\newcommand{\RED}[1]{#1}

\newcommand\TODO[1]{{\color{red}{TODO: #1}}}
% \newcommand\coord[1]{\bm{#1}}

% \newcommand\TODO[1]{#1}
\newcommand\UPDATE[1]{{\color{blue}{#1}}}
\newcommand\REVIEW[1]{#1}
\newcommand\operator[1]{\textrm{#1}}
% \newcommand\UPDATE[1]{#1}
\newcommand\SOURCE[1]{{\color{green}{(from: #1)}}}


% LSI masks definitions:
\newcommand*{\maskname}{central instance~}
\newcommand*{\sparseBr}{sparse-neighborhood~}
\newcommand*{\denseBr}{dense-neighborhood~}
\newcommand*{\encBr}{encoded-neighborhood~}
\newcommand*{\maskDec}{central-mask-decoder~}
\newcommand*{\evidW}{w}
\newcommand\coord{\vec}
% \newcommand*{\coord}{\bm}

% GASP definitions:
\newcommand*{\cost}{w}%
\newcommand*{\interact}{\mathcal{W}}
\newcommand*{\NBE}{\Gamma} % number of edges in a boundary 
\newcommand*{\nBE}{\gamma}
\newcommand*{\algname}{GASP}
\newcommand*{\treeHeight}{\interact{}_{T}}
\newcolumntype{M}[1]{>{\centering\arraybackslash}m{#1}}
\newcolumntype{R}[1]{>{\raggedleft\arraybackslash}m{#1}} 
\newcolumntype{L}[1]{>{\raggedright\arraybackslash}m{#1}} 
 \newcolumntype{?}{!{\vrule width 0.3em}}

\usetikzlibrary{matrix,positioning,calc}
\tikzset{line/.style ={draw, rounded corners=2pt, line width=1pt}}

\newcommand\tikzmark[1]{%
\tikz[remember picture]  \node[inner sep=0,outer sep=0] (#1){};%
}



% MWS definitions:
\newcommand*{\QEDA}{\hfill\ensuremath{\blacksquare}}%
\newcommand*{\intweight}{\mathfrak{r}}%
\newcommand*{\cutindicator}{a}%
\newcommand*{\mccutindicator}{y}%
\newcommand*{\subproblem}{S}%
\newcommand*{\Ainit}{\tilde{A}}%
\newcommand*{\intSC}{ISC}%
\newcommand*{\mcw}{{\tilde w}}%
\newcommand*{\gt}[1]{\stackrel{\smash{*}\rule{0pt}{-1.ex}}{#1}}%


\usepackage{thmtools}
\usepackage{thm-restate}
\usepackage{amsthm}
\newtheorem{theorem}{Theorem}[section]
\newtheorem{prop}{Proposition}[section]
\newtheorem{observation}{Observation}[section]
\newtheorem{corollary}{Corollary}[theorem]
\newtheorem{lemma}[theorem]{Lemma}
\newtheorem{property}{Property}[section]
\newtheorem{definition}{Definition}[section]



\usepackage{tabularx}
\usepackage{multirow}
\usepackage{makecell}
\usepackage{textcomp}
\usepackage{booktabs}
\usepackage{marvosym}
% \usepackage[labelformat=simple]{subcaption}
\newcolumntype{M}[1]{>{\centering\arraybackslash}m{#1}}
\newcolumntype{R}[1]{>{\raggedleft\arraybackslash}m{#1}} 
\newcolumntype{L}[1]{>{\raggedright\arraybackslash}m{#1}} 
\newcolumntype{?}{!{\vrule width 0.3em}}


\usepackage{algorithm}% http://ctan.org/pkg/algorithm
\PassOptionsToPackage{noend}{algpseudocode}% comment out if want end's to show
\usepackage{algpseudocode}% http://ctan.org/pkg/algorithmicx
\algrenewcommand\algorithmicindent{0.8em}


\makeatletter
\newcommand*{\skipnumber}[2][1]{%
   {\renewcommand*{\alglinenumber}[1]{}\State #2}%
   \addtocounter{ALG@line}{-#1}}
\renewcommand{\ALG@beginalgorithmic}{\small}
\makeatother

% \usepackage[dvipsnames]{xcolor}
\usepackage{hyperref}
