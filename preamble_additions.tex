% !TEX root = ./main.tex
\usepackage{wrapfig}
\usepackage{amsfonts}
\usepackage{amsmath,amssymb}
\usepackage{subcaption}
\usepackage{epsfig}
\usepackage{tikz}
\usepackage{etoolbox}
\usepackage{bbm}
\usepackage{mathrsfs}
\captionsetup{compatibility=false}
% \usepackage[caption=false]{subfig}

\DeclareMathOperator*{\argmax}{arg\,max}
\DeclareMathOperator*{\argmin}{arg\,min}
\usepackage{mathrsfs}
\newcommand{\RED}[1]{#1}

\newcommand\TODO[1]{{\color{red}{TODO: #1}}}
% \newcommand\coord[1]{\bm{#1}}

% \newcommand\TODO[1]{#1}
\newcommand\UPDATE[1]{{\color{blue}{#1}}}
\newcommand\REVIEW[1]{#1}
\newcommand\operator[1]{\textrm{#1}}
% \newcommand\UPDATE[1]{#1}
\newcommand\SOURCE[1]{{\color{green}{(from: #1)}}}


% LSI masks definitions:
\newcommand*{\maskname}{central instance~}
\newcommand*{\sparseBr}{sparse-neighborhood~}
\newcommand*{\denseBr}{dense-neighborhood~}
\newcommand*{\encBr}{encoded-neighborhood~}
\newcommand*{\maskDec}{central-mask-decoder~}
\newcommand*{\evidW}{w}
\newcommand\coord{\vec}
% \newcommand*{\coord}{\bm}

% GASP definitions:
\newcommand*{\cost}{w}%
\newcommand*{\interact}{\mathcal{W}}
\newcommand*{\NBE}{\Gamma} % number of edges in a boundary 
\newcommand*{\nBE}{\gamma}
\newcommand*{\algname}{GASP}
\newcommand*{\treeHeight}{\interact{}_{T}}
\newcolumntype{M}[1]{>{\centering\arraybackslash}m{#1}}
\newcolumntype{R}[1]{>{\raggedleft\arraybackslash}m{#1}} 
\newcolumntype{L}[1]{>{\raggedright\arraybackslash}m{#1}} 
 \newcolumntype{?}{!{\vrule width 0.3em}}

\usetikzlibrary{matrix,positioning,calc}
\tikzset{line/.style ={draw, rounded corners=2pt, line width=1pt}}

\newcommand\tikzmark[1]{%
\tikz[remember picture]  \node[inner sep=0,outer sep=0] (#1){};%
}



% MWS definitions:
\newcommand*{\QEDA}{\hfill\ensuremath{\blacksquare}}%
\newcommand*{\intweight}{\mathfrak{r}}%
\newcommand*{\cutindicator}{a}%
\newcommand*{\mccutindicator}{y}%
\newcommand*{\subproblem}{S}%
\newcommand*{\Ainit}{\tilde{A}}%
\newcommand*{\intSC}{ISC}%
\newcommand*{\mcw}{{\tilde w}}%
\newcommand*{\gt}[1]{\stackrel{\smash{*}\rule{0pt}{-1.ex}}{#1}}%


\usepackage{thmtools}
\usepackage{thm-restate}
\usepackage{amsthm}
\newtheorem{theorem}{Theorem}[section]
\newtheorem{prop}{Proposition}[section]
\newtheorem{observation}{Observation}[section]
\newtheorem{corollary}{Corollary}[theorem]
\newtheorem{lemma}[theorem]{Lemma}
\newtheorem{property}{Property}[section]
\newtheorem{definition}{Definition}[section]



\usepackage{tabularx}
\usepackage{multirow}
\usepackage{makecell}
\usepackage{textcomp}
\usepackage{booktabs}
\usepackage{marvosym}
% \usepackage[labelformat=simple]{subcaption}
\newcolumntype{M}[1]{>{\centering\arraybackslash}m{#1}}
\newcolumntype{R}[1]{>{\raggedleft\arraybackslash}m{#1}} 
\newcolumntype{L}[1]{>{\raggedright\arraybackslash}m{#1}} 
\newcolumntype{?}{!{\vrule width 0.3em}}


\usepackage{algorithm}% http://ctan.org/pkg/algorithm
\PassOptionsToPackage{noend}{algpseudocode}% comment out if want end's to show
\usepackage{algpseudocode}% http://ctan.org/pkg/algorithmicx
\algrenewcommand\algorithmicindent{0.8em}

% Modifications for algorithms:
\makeatletter
\newcommand*{\skipnumber}[2][1]{%
   {\renewcommand*{\alglinenumber}[1]{}\State #2}%
   \addtocounter{ALG@line}{-#1}}
\renewcommand{\ALG@beginalgorithmic}{\small}
\makeatother

% Add possible \algcomment{} with descirption below a pseudocode:
\makeatletter
\AfterEndEnvironment{algorithm}{\let\@algcomment\relax}
\AtEndEnvironment{algorithm}{\kern2pt\hrule\relax\vskip3pt\@algcomment}
\let\@algcomment\relax
\newcommand\algcomment[1]{\def\@algcomment{\footnotesize#1}}
\renewcommand\fs@ruled{\def\@fs@cfont{\bfseries}\let\@fs@capt\floatc@ruled
  \def\@fs@pre{\hrule height.8pt depth0pt \kern2pt}%
  \def\@fs@post{}%
  \def\@fs@mid{\kern2pt\hrule\kern2pt}%
  \let\@fs@iftopcapt\iftrue}
\makeatother

% \errorcontextlines\maxdimen

%%%%%%%%%%%%%%%%%%%%%%%%%

% \errorcontextlines\maxdimen

% % begin vertical rule patch for algorithmicx (http://tex.stackexchange.com/questions/144840/vertical-loop-block-lines-in-algorithmicx-with-noend-option)
% % note that some of the packages above are also needed
% \newcommand{\ALGtikzmarkcolor}{black}% customise this, if you want
% \newcommand{\ALGtikzmarkextraindent}{4pt}% customise this, if you want
% \newcommand{\ALGtikzmarkverticaloffsetstart}{-.5ex}% customise this, if you want
% \newcommand{\ALGtikzmarkverticaloffsetend}{-.5ex}% customise this, if you want
% \makeatletter
% \newcounter{ALG@tikzmark@tempcnta}

% \newcommand\ALG@tikzmark@start{%
%     \global\let\ALG@tikzmark@last\ALG@tikzmark@starttext%
%     \expandafter\edef\csname ALG@tikzmark@\theALG@nested\endcsname{\theALG@tikzmark@tempcnta}%
%     \tikzmark{ALG@tikzmark@start@\csname ALG@tikzmark@\theALG@nested\endcsname}%
%     \addtocounter{ALG@tikzmark@tempcnta}{1}%
% }

% \def\ALG@tikzmark@starttext{start}
% \newcommand\ALG@tikzmark@end{%
%     \ifx\ALG@tikzmark@last\ALG@tikzmark@starttext
%         % ignore this, the block was opened then closed directly without any other blocks in between (so just a \State basically)
%         % don't draw a vertical line here
%     \else
%         \tikzmark{ALG@tikzmark@end@\csname ALG@tikzmark@\theALG@nested\endcsname}%
%         \tikz[overlay,remember picture] \draw[\ALGtikzmarkcolor] let \p{S}=($(pic cs:ALG@tikzmark@start@\csname ALG@tikzmark@\theALG@nested\endcsname)+(\ALGtikzmarkextraindent,\ALGtikzmarkverticaloffsetstart)$), \p{E}=($(pic cs:ALG@tikzmark@end@\csname ALG@tikzmark@\theALG@nested\endcsname)+(\ALGtikzmarkextraindent,\ALGtikzmarkverticaloffsetend)$) in (\x{S},\y{S})--(\x{S},\y{E});%
%     \fi
%     \gdef\ALG@tikzmark@last{end}%
% }



% % the following line injects our new tikzmarking code
% \apptocmd{\ALG@beginblock}{\ALG@tikzmark@start}{}{\errmessage{failed to patch}}
% \pretocmd{\ALG@endblock}{\ALG@tikzmark@end}{}{\errmessage{failed to patch}}
% \makeatother
% % end vertical rule patch for algorithmicx
% %%%%%%%%%%%%%%%%%%%%%%%%%% 



% \usepackage[dvipsnames]{xcolor}
\usepackage{hyperref}
