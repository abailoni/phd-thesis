%!TEX root = ../../main.tex

\section{Conclusion} \label{Conclusion}
We have presented a fast algorithm for the clustering of graphs with both attractive and repulsive edges. The ability to consider both
\REVIEW{gives a valid alternative to other popular graph partitioning algorithms that rely on a stopping criterion or seeds.}
 % obviates the need for the kind of stopping criterion or even seeds that all popular algorithms except for multicut / correlation clustering need. 
 The proposed method has low computational complexity in imitation of its close relative, Kruskal's algorithm. We have shown which objective this algorithm optimizes exactly, and that this objective emerges as a specific case of the multicut objective. \RED{It is possible that recent interesting work \cite{lange2018partial} on partial optimal solutions may open an avenue for an alternative proof.
%An alternative approach for our proof may be derived from the recent work by \cite{lange2018partial} on partial optimal solutions. 
% We  showing that every greedy choice can be shown to be a partial optimal solution. 
%every edge fixed the the Mutex Watershed is part of one optimal solution.%, where they show that repulsive / attractive edges are cut / not cut in the optimal multicut solution when the edge's absolute weight is larger than the sum of absolute weights of edges along a given cutntro.
}
% facilitate an alternative method to prove this property,  showing that repulsive / attractive edges are cut / not cut in the optimal multicut solution when the edge's absolute weight is larger than the sum of absolute weights of edges along a given cut.}
Finally, we have found that the proposed algorithm, when presented with informative edge costs from a good neural network, outperforms all known methods on a competitive bioimage partitioning benchmark, including methods that operate on the very same network predictions. 


% \RED{We prove using dynamic programming that -- in this limit -- the optimal partitioning can be determined iteratively in order of absolute edge weight. Recent analysis of the multicut objective \cite{lange2018partial} suggests an alternative method to prove this property,  showing that repulsive / attractive edges are cut / not cut in the optimal multicut solution when the edge's absolute weight is larger than the sum of absolute weights of edges along a given cut.}


% In future work we want to generalize our algorithm to semantic instance segmentation commonly found in natural image segmentation challenges \cite{Cordts2016Cityscapes,lin2014microsoft,mottaghi_cvpr14} and use it for end-to-end learning.
