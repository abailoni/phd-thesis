%!TEX root = ../main.tex
%%%%%%%%% ABSTRACT

% for Computer Society papers, we must declare the abstract and index terms
% PRIOR to the title within the \IEEEtitleabstractindextext IEEEtran
% command as these need to go into the title area created by \maketitle.
% As a general rule, do not put math, special symbols or citations
% in the abstract or keywords.
\IEEEtitleabstractindextext{%
\begin{abstract}
Image partitioning, or segmentation without semantics, is the task of decomposing an image into distinct segments, or equivalently to detect closed contours. Most prior work either requires seeds, one per segment; or a threshold; or formulates the task as multicut / correlation clustering, an NP-hard problem. Here, we propose an \REVIEW{efficient} algorithm for \REVIEW{graph partitioning}, the ``Mutex Watershed''. Unlike seeded watershed, the algorithm can accommodate not only attractive but also repulsive cues, allowing it to find a previously \emph{unspecified} number of segments without the need for explicit seeds or a tunable threshold. We also prove that this simple algorithm solves to global optimality an objective function that is intimately related to the multicut / correlation clustering integer linear programming formulation. 
The algorithm is deterministic, very simple to implement, and has empirically linearithmic complexity. 
%It can be related to with empirically linearithmic complexity. %Just like multicut / correlation clustering or modularity clustering, it determines an optimal number of segments automatically; but unlike these criteria, ours exhibits matroid structure and can be optimized globally by a simple greedy algorithm akin to a minimal spanning tree computation. 
%The algorithm itself, which we dub ``Mutex Watershed'', is closely related to a minimal spanning tree computation. It is deterministic and easy to implement. 
When presented with short-range attractive and long-range repulsive cues from a deep neural network, the Mutex Watershed gives the best results currently known for the %define the state-of-the-art in the 
competitive ISBI 2012 EM segmentation benchmark. %These results are also better than those obtained from other recently proposed clustering strategies operating on the very same network outputs.  %, besting highly engineered pipelines with NP-hard correlation clustering postprocessing. 
\end{abstract}

% Note that keywords are not normally used for peerreview papers.
\begin{IEEEkeywords}
Image segmentation, partitioning algorithms, greedy algorithms, optimization, integer linear programming, machine learning, convolutional neural networks. 
\end{IEEEkeywords}}


% make the title area
\maketitle


% To allow for easy dual compilation without having to reenter the
% abstract/keywords data, the \IEEEtitleabstractindextext text will
% not be used in maketitle, but will appear (i.e., to be "transported")
% here as \IEEEdisplaynontitleabstractindextext when the compsoc 
% or transmag modes are not selected <OR> if conference mode is selected 
% - because all conference papers position the abstract like regular
% papers do.
\IEEEdisplaynontitleabstractindextext
% \IEEEdisplaynontitleabstractindextext has no effect when using
% compsoc or transmag under a non-conference mode.



% For peer review papers, you can put extra information on the cover
% page as needed:
% \ifCLASSOPTIONpeerreview
% \begin{center} \bfseries EDICS Category: 3-BBND \end{center}
% \fi
%
% For peerreview papers, this IEEEtran command inserts a page break and
% creates the second title. It will be ignored for other modes.
\IEEEpeerreviewmaketitle


