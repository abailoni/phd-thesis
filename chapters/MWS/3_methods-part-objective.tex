%!TEX root = ../main.tex
\clearpage

\section{Objective and Path Inequalities\label{subsec:Clustering-path}}

\begin{figure}[t]
\centering
\includegraphics[width=\linewidth]{./images/path_examples2.pdf}
\label{fig:path_examples}
\end{figure}

% In this section we want to define an objective function that is optimized by the Mutex Watershed algorithm. 

Many clustering schemes~(e.g. agglomerative clustering) define a measure of attraction and merge clusters until an attraction threshold is reached.
% Therefore, clusters have a high within clusters and minimized between clusters.
We eliminate the need for such a threshold by introducing a second measure of repulsion. Thereby, a natural clustering definition emerges, where the attraction within clusters is higher than the repulsion, and the repulsion between clusters is higher than the attraction.

\noindent To formalize this in the spirit of the watershed algorithm we first define attraction and repulsion between clusters, utilizing specialized maximin paths. 
\begin{definition} For a graph $\mathcal{G}(V,E^{+}\cup E^{-})$ and cut $\partial S$, the \textbf{set of attractive paths }$\Pi_{v_{i}\rightarrow v_{j}}^{+}$ (\textbf{connecting $v_{i} \in V$ and $v_{j} \in V$}) is the set of all paths, such that $\pi^+ \in \Pi_{v_{i}\rightarrow v_{j}}^{+}$ comprises \textit{only attractive edges} $\pi^+ \subseteq E^{+}$ and crosses the cut up to one time 
\begin{equation}
|\pi^+ \cap \partial S| \le 1\label{eq:attr_def_1}
\end{equation}
\end{definition}
Attractive paths can connect nodes within a cluster($|\pi^+ \cap \partial S| = 0$) and nodes of neighboring clusters($|\pi^+ \cap \partial S| = 1$), but there are no attractive paths between non-neighboring clusters.
\begin{definition} For a graph $\mathcal{G}(V,E^{+}\cup E^{-})$ and cut $\partial S$, the \textbf{set of repulsive paths }$\Pi_{v_{i}\rightarrow v_{j}}^{-}$ (\textbf{connecting $v_{i} \in V$ and $v_{j} \in V$}) is the set of all paths in $\mathcal{G}$, such that every path $\pi^- \in \Pi_{v_{i}\rightarrow v_{j}}^{-}$ contains \textit{exactly one repulsive edge} \begin{equation}\left|\pi^-\cap E^{-}\right|=1\label{eq:rep_def_2}\end{equation} that crosses the cut \begin{equation}\left|\pi^-\cap (\partial S \cap E^{-})\right|=1\label{eq:rep_def_2}\end{equation} and does not cross the cut otherwise \begin{equation}
|\pi^- \cap (\partial S \cap E^{+})| = 0\label{eq:attr_def_2}
\end{equation}
\end{definition}

Repulsive paths connect nodes of neighboring clusters and cross the cut using a repulsive edge $\in E^-$.
Utilizing these paths, we can now define measures of  attraction and repulsion:
\begin{definition}[Attraction] The attraction $w_{v_{i}\rightarrow v_{j}}^{+}$ between $v_i\in V$ and $v_j \in V$ is defined by the maximin attractive path:
\begin{equation}
w_{v_{i}\rightarrow v_{j}}^{+}\equiv\begin{cases}
\underset{\pi \in \Pi_{v_{i}\rightarrow v_{j}}^{+}}{\max} \underset{e \in \pi}{\min} \quad w(e) & \operator{if}\quad\Pi_{v_{i}\rightarrow v_{j}}^{+}\neq\emptyset\\
\quad\quad\quad-\epsilon & \operator{else}
\end{cases}\label{eq:maximin_weight_a}
\end{equation}, with $\epsilon > 0$.
\end{definition}


\begin{definition}[Repulsion] The repulsion $w_{v_{i}\rightarrow v_{j}}^{-}$ between $v_i\in V$ and $v_j \in V$ is defined by the maximin repulsive path:
\begin{equation}
w_{v_{i}\rightarrow v_{j}}^{-}\equiv\begin{cases}
\underset{\pi \in \Pi_{v_{i}\rightarrow v_{j}}^{-}}{\max} \underset{e \in \pi}{\min} \quad w(e) & \operator{if}\quad\Pi_{v_{i}\rightarrow v_{j}}^{-}\neq\emptyset\\
\quad\quad\quad 0 & \operator{else}
\end{cases}\label{eq:maximin_weight_b}
\end{equation}
\end{definition}
\noindent Note, that if clusters are not connected~(no attractive nor repulsive paths exist), $w^+=-\epsilon < 0 = w^-$ breaks the tie in favor of repulsion/splitting the clusters.

\noindent With these measures, we can define the Mutex Watershed Cut properties.

\begin{definition} [Mutex Watershed Cut / Path inequalities] \label{def:path_ineq}\\
For a weighted graph ${\mathcal{G}=(V,E^{+}\cup E^{-}, W)}$ we say that a cut $\partial S$ of $\mathcal{G}$ is a Mutex Watershed Cut if , % with corresponding partition $S\equiv\left\{ C_{1},...,C_{K}\right\}$ into $K$  connected subsets
 the following set of path inequalities are fulfilled: \\
 For all repulsive edges $e^-$ inside a cluster the attraction is higher than the repulsive edge weight.
\begin{equation}
\forall(v_{i},v_{j})=e^{-}\in E^{-}\setminus \partial S: \quad \quad w_{v_{i}\rightarrow v_{j}}^{+} > w(e^{-}),\label{eq:path_rel_1}
\end{equation}
 For all attractive edges $e^+$ connecting two cluster the repulsion %of the incident vertices 
 is higher than the attractive edge weight.
\begin{equation}
\forall(v_{i},v_{j})=e^{+}\in E^{+}\cap \partial S \quad \quad w_{v_{i}\rightarrow v_{j}}^{-}>w(e^{+}).\label{eq:path_rel_2}
\end{equation}
\end{definition}


\noindent Finally, we propose

\noindent \textbf{Theorem 1 (Mutex Watershed Algorithm finds Mutex Watershed Cut)}
\emph{On a weighted graph ${\mathcal{G}=(V,E^{+}\cup E^{-}, W)}$ with unique weights, the Mutex Watershed Cut is unique and found by \autoref{WS_algo_code}.}

For a detailed proof we of Theorem 1, we refer the reader to the supplementary material.

