% !TEX root = ../../main.tex

\section{Related Work} \label{2_rel_work}

\noindent In the original watershed algorithm \cite{vincent1991watersheds,Beucher-Lantu-79}, seeds were automatically placed at all local minima of the boundary map. Unfortunately, this leads to severe over-segmentation. Defining better seeds has been a recurring theme of watershed research ever since. The simplest solution is offered by the seeded watershed algorithm \cite{beucher1992morphological}: It relies on an oracle (an external algorithm or a human) to provide seeds and assigns each pixel to its nearest seed in terms of minimax path distance.

\REVIEW{In the absence of an oracle, many automatic methods for seed selection have been proposed in the last decades with applications in the fields of medicine and biology. Many of these approaches rely on edge feature extraction and edge detection like gradient calculation \cite{pohle2001segmentation,alattar2010myocardial}. Other types of methods generate seeds by first performing feature extraction \cite{poonguzhali2006complete,wu2008texture}, whereas others first extract region of interests and then place seeds inside these regions by using thresholding \cite{al2014computer}, binarization \cite{shan2008novel}, $k$-means \cite{mubarak2012hybrid} or other strategies \cite{abdelsamea2011enhancement,al2014breast}. 
% In this work seeds are interpreted as mutual exclusion constraints and the presented algorithm, that selects these constraints to form clusters can therefore be seen as a complementary approach to seed selection.
}

\REVIEW{In applications where the number of regions is hard to estimate, simple automatic seed selection methods, e.g.\ defining seeds by connected regions of low boundary probability, don't work: The segmentation quality is usually insufficient because multiple seeds are in the same region and/or seeds leak through the boundary.} Thus, in these cases seed selection may be biased towards over-seg\-men\-ta\-tion (with seeding at all minima being the extreme case). The watershed algorithm then produces superpixels that are merged into final regions by more or less elaborate postprocessing. This works better than using watersheds alone because it exploits the larger context afforded by superpixel adjacency graphs. Many criteria have been proposed to identify the regions to be preserved during merging, e.g.\ region dynamics \cite{grimaud_92_watershed-dynamics}, the waterfall transform \cite{beucher1994watershed}, extinction values \cite{vachier1995extinction}, region saliency \cite{najman1996geodesic}, and $(\alpha,\omega)$-connected components \cite{soille_08_hierarchical-image-decomposition}. A merging process controlled by criteria like these can be iterated to produce a hierarchy of segmentations where important regions survive to the next level. Variants of such hierarchical watersheds are reviewed and evaluated in \cite{perret_17_hierarchical-watersheds}.

These results highlight the close connection of watersheds to hierarchical clustering and minimum spanning trees/forests \cite{meyer1999morphological,najman_11_ultrametric-watersheds}, which inspired novel merging strategies and termination criteria. For example, \cite{salembier_00_binary-partition-tree} simply terminated hierarchical merging by fixing the number of surviving regions beforehand. \cite{malmberg2011generalized} incorporate predefined sets of generalized merge constraints into the clustering algorithm. Graph-based segmentation according to \cite{felzenszwalb_04_graph-based-image-segmentation} defines a measure of quality for the current regions and stops when the merge costs would exceed this measure. Ultrametric contour maps \cite{arbelaez_11_gpb} combine the gPb (global probability of boundary) edge detector with an oriented watershed transform. Superpixels are agglomerated until the ultrametric distance between the resulting regions exceeds a learned threshold. An optimization perspective is taken in \cite{kiran_13_hierarchical-cuts,guigues2006scale}, which introduces $h$-increasing energy functions and builds the hierarchy incrementally such that merge decisions greedily minimize the energy. The authors prove that the optimal cut corresponds to a different unique segmentation for every value of a free regularization parameter.

\REVIEW{An important line of research is given by partitioning of graphs with both attractive and repulsive edges \cite{keuper2016multi}.} Solutions that optimally balance attraction and repulsion do not require external stopping criteria such as predefined number of regions or seeds. This generalization leads to the NP-hard problem of correlation clustering or (synonymous) multicut (MC) partitioning. Fortunately, modern integer linear programming solvers in combination with incremental constraint generation can solve problem instances of considerable size \cite{andres_12_globally}, and good approximations exist for even larger problems \cite{yarkony2012fast,pape2017solving}
Reminiscent of strict minimizers \cite{Levi} with minimal $L_\infty$-norm solution, our work solves the multicut objective optimally when all graph weights are raised to a large power.


% In the limit that transforms the multicut objective into the Mutex Watershed objective, the strongest edge along a given cut is dominant by construction; an observation that
% hints at a different approach to proving the optimality of the Mutex Watershed for its objective.

% When we consider the MC as an $L_1$ minimization problem the $p \rightarrow \infty$


% . In geometric optimization strict minimizers, are the limit of $L_p$-norm minimizers for $p \rightarrow \infty$,  \cite{Levi}. We show that mc can be efficitently solved in the $p \rightarrow \infty$ limit.


% $p$-norm, for $p \rightarrow \infty$ $L_1$-norm of all cut energies minimal $L_{\infty}$-norm solution similar to strict minimizer

Related to the proposed method, the greedy additive edge contraction (GAEC) \cite{keuper2015efficient} heuristic for the multicut also sequentially merges regions, but we handle attractive and repulsive interactions separately and define edge strength between clusters by a maximum instead of an additive rule.
The greedy fixation algorithm introduced in \cite{levinkov2017comparative} is closely related to the proposed method; it sorts attractive and repulsive edges by their absolute
weight, merges nodes connected by attractive edges and introduces no-merge constraints
for repulsive edges. However, similar to GAEC, it defines edge strength by an additive rule,
which increases the algorithm's runtime complexity compared to the presented Mutex Watershed. Also, it is not yet known what objective the algorithm optimizes globally, if any.

Another beneficial extension is the introduction of additional long-range edges. The strength of such edges can often be estimated with greater certainty than is achievable for the local edges used by watersheds on standard 4- or 8-connected pixel graphs. Such repulsive long-range edges have been used in \cite{zhang_14_cell} to represent object diameter constraints, which is still an MC-type problem. When long-range edges are also allowed to be attractive, the problem turns into the more complicated lifted multicut (LMC) \cite{horvnakova2017analysis}. Realistic problem sizes can only be solved approximately \cite{keuper2015efficient,beier2016efficient}, but watershed superpixels followed by LMC postprocessing achieve state-of-the-art results on important benchmarks \cite{beier2017multicut}. Long-range edges are also used in \cite{lee2017superhuman}, as side losses for the boundary detection convolutional neural network~(CNN); but they are not used explicitly in any downstream inference.

In general, striking progress in watershed-based segmentation has been achieved by learning boundary maps with CNNs. This is nicely illustrated by the evolution of neurosegmentation for connectomics, an important field we also address in the experimental section. CNNs were introduced to this application in \cite{jain2007supervised} and became, in much refined form \cite{ciresan_12_deep-em-segmentation}, the winning entry of the ISBI 2012 Neuro-Segmentation Challenge \cite{isbi2012challenge}. Boundary maps and superpixels were further improved by progress in CNN architectures and data augmentation methods, using U-Nets \cite{ronneberger_15_u-net}, FusionNets \cite{quan2016fusionnet} or inception modules \cite{beier2017multicut}. Subsequent postprocessing with the GALA algorithm \cite{GALA,knowles2016rhoananet}, conditional random fields \cite{uzunbacs_14_optree} or the lifted multicut \cite{beier2017multicut} pushed the envelope of final segmentation quality. MaskExtend \cite{meirovitch2016multi} applied CNNs to both boundary map prediction and superpixel merging, while flood-filling networks \cite{floodfill} eliminated superpixels altogether by training a recurrent neural network to perform region growing one region at a time.

Most networks mentioned so far learn boundary maps on pixels, but learning works equally well for edge-based watersheds, as was demonstrated in \cite{zlateski2015image,parag2017anisotropic} using edge weights generated with a CNN \cite{turaga2010convolutional,MALIS}. Tayloring the learning objective to the needs of the watershed algorithm by penalizing critical edges along minimax paths \cite{MALIS} or end-to-end training of edge weights and region growing \cite{wolf2017learned} improved results yet again.

Outside of connectomics, \cite{bai2016deep_watershed} obtained superior boundary maps from CNNs by learning not just boundary strength, but also its gradient direction. Holistically-nested edge detection \cite{xie2015holistically,kokkinos2015pushing} couples the CNN loss at multiple resolutions using deep supervision and is successfully used as a basis for watershed segmentation of medical images in \cite{cai2016pancreas}. 

We adopt important ideas from this prior work (hierarchical single-linkage clustering, attractive and repulsive interactions, long-range edges, and CNN-based learning). The proposed efficient segmentation framework can be interpreted as a generalization of \cite{malmberg2011generalized}, because we also allow for soft repulsive interactions (which can be overridden by strong attractive edges), and constraints are generated on-the-fly.
