%!TEX root = ../main.tex

%\cleardoublepage

\section{Efficient implementation of the Mutex Watershed Algorithm}

% In our application, each vertex corresponds to one pixel in the image to be segmented. Two vertices may have no edge connecting them; or an attractive edge $e\in  E^+$; or a repulsive edge  $e\in  E^-$; or two edges at the same time, one attractive and one repulsive. Edges can be either {\it local/short-range} (when connecting two pixels that are immediately adjacent in the image) or {\it long-range}. 
% Our algorithm ``activates'' attractive edges to merge and introduces mutual exclusion relations by ``activating'' repulsive edges. Given the ``active'' subset $A^+ \subseteq E^+$ of attractive edges, clusters are defined via the ``connected'' predicate: 
% The mutex watershed algorithm, defined in \autoref{3_3_MWS}, maintains disjunct active sets $A^+ \subseteq E^+$, $A^- \subseteq E^-$, $A^+ \cap A^- = \emptyset$, that encode merges and mutual exclusion constraints, respectively. 

In this section we propose an efficient implementation of the MWS, derive its theoretical average runtime complexity and show that in practice the MWS scales with $\mathcal{O}(E \log E)$.

\subsection*{Notation:}
Let us explicitly define clusters using the ``connected'' predicate $\forall i,j \in V$: 
% \begin{equation}
% \forall i,j\in V: \,\,
% \Pi_{i \rightarrow j}  =  \{ \mathrm{path\,} \pi\,\mathrm{from}\,i\,\mathrm{to}\,j\,\,\mathrm{s.t.}\,\pi\subseteq E\}
% \end{equation}
\begin{align*}
\mathrm{connected}(i, j | A)\,\,:= \,\, & \mathbbm{I}\{\exists \textrm{ path } \pi \subseteq A^+ \textrm{ from }i\textrm{ to }j\}\\ %&\textrm{ s.t. }\pi \subseteq A^+ \subseteq E^+ 
% & \forall i,j \in V\\
\mathrm{cluster}(i | A) = \,\, &  \{i\} \cup \{j: \mathrm{connected}(i,j | A) \}
\end{align*}
Conversely, the active subset $A^-\subseteq E^-$ of repulsive edges defines mutual exclusion relations by using the following predicate:
\begin{align*}
\mathrm{mutex}(i, j | A) \,\,:= \,\, & \mathbbm{I}\{ \exists \textrm{ path } \pi \subseteq A \textrm{ from }i\textrm{ to }j \textrm{ s.t.  }|\pi \cap A^-| = 1\} \nonumber
\end{align*}
To implement the MWS efficiently we reformulate the algorithm in terms of the predicates $\mathrm{connected}$ and $\mathrm{mutex}$ using the following equalities (for an arbitrary set 
% $A \in \{\tilde{A} \subseteq E \;|\; {C}_{0/1}(\tilde{A}) =  \emptyset\}$)
$A \subseteq E$ with ${C}_{0/1}(A) =  \emptyset$)
\begin{align*}
% \begin{split}
\forall (i, j) \in E^+\setminus A:\quad &
\!\!\!\mathrm{connected}(i, j | A)\!\!\!\!\!& \Leftrightarrow & \,\, {C}_{0}(A\cup \{(i, j)\}) \neq \emptyset\\
&\!\!\!\mathrm{mutex}(i, j | A)\!\!\!\!\!&\Leftrightarrow & \,\, {C}_{1}(A\cup \{(i, j)\}) \neq \emptyset\\
% \intertext{and}
% \end{split}
% \end{align}
% \begin{align}
\forall (i, j) \in E^-\setminus A:\quad&
\!\!\!\mathrm{connected}(i, j | A)\!\!\!\!\!&\Leftrightarrow & \,\, {C}_{1}(A\cup \{(i, j)\}) \neq \emptyset
\end{align*} to rewrite the Mutex Watershed Algorithm as Algorithm \ref{algo_code_efficient}. In the following analysis we omit the dependence on $A$ in all predicates to improve readability.

% \begin{minipage}[t]{0.35\linewidth}
% algorithm 1 here ??????
% \end{minipage}%
% \begin{minipage}[t]{0.65\linewidth}
\begin{algorithm}[h]
% \centering
% \parbox[t]{.7\linewidth}{
 \hrulefill \\
   \textbf{Input: }
     weighted graph $\mathcal{G}(V,  E^+\cup E^-, W^+\cup W^-)$\;
      \textbf{Output: }
     clusters defined by active set $A^+$\;
  \textbf{Initialization: }
     $A^+ = \emptyset $; \quad
     $A^- = \emptyset $\;
 \For{$(i,j) = e \in (E^+\cup \! E^-) $ {\rm in descending order of } $W^+ \cup W^-$} 
 {
   \eIf{$e \in E^+$}
   {
   \If{\rm {\bf not} $\mathrm{connected}$($i, j$) \textbf{and not} $\mathrm{mutex}$($i,j$) }  
    {
          $\mathrm{merge}(i, j)$: $A^+ \leftarrow A^+ \cup e$\;
          \Comment{\parbox[t]{.65\linewidth}{ merge $i$ and $j$ and inherit the mutex \\constraints of the parent clusters}}
          % \Comment{merge $i$ and $j$}\\
               % \Comment{and inherit the mutex constraints of the parent clusters}
    }
   }{
    \If{\rm {\bf not} $\mathrm{connected}$($i, j$)}
      {
          $\mathrm{addmutex}(i, j)$: $A^- \leftarrow A^- \cup e$\;
          \Comment{\parbox[t]{.65\linewidth}{add mutex constraint between $i$ and $j$}}
      }
   }
 } \vspace{-6pt}\hrulefill
 \vspace{6pt}
 \caption{Efficient implementation of Mutex Watershed. The $\mathrm{connected}$ predicate can be efficiently evaluated.}
 \label{algo_code_efficient}
% }
\end{algorithm}
% \end{minipage}

% Observe, that two nodes that are already connected via a path with one repulsive edge are mutually exclusive and can not be merged, since this would create a cycle $\in \mathcal{C}_1$.
