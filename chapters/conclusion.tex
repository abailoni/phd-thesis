%!TEX root = ../main.tex


\chapter*{Conclusions}
\addcontentsline{toc}{chapter}{Conclusions} 
In this thesis, we have developed new graph-based instance segmentation methods and applied them to the task of automated segmentation of large neural tissue 3D image volumes. Previous methods often relied on multi-step post-processing pipelines based on superpixels, which usually required the user to spend time tuning task-dependent hyper-parameters. 
On the other hand, the simple graph partitioning algorithms presented in this thesis are parameter-free. Some of them, when combined with informative edge weights from a deep neural network, achieve state-of-the-art accuracies on two neuron segmentation challenges.

In Chapter ~\ref{chapter:MWS}, we have introduced a new efficient algorithm, the Mutex Watershed, for the clustering of graphs with both attractive and repulsive edge weights. The proposed algorithm has low computational complexity and is closely related to the Kruskal's algorithm for minimum spanning tree computation \cite{kruskal1956shortest}.
We prove that this algorithm finds the global optimum of an objective function, and that this objective is closely related to the multicut optimization problem as well as the power watershed framework.
The Mutex Watershed algorithm achieved state-of-the-art results on the ISBI neuron segmentation challenge, and, after our results were published, it also became part of other two state-of-the-art segmentation pipelines used for neuron segmentation \cite{hirsch2020patchperpix,lee2019learning}.

In Chapter \ref{chapter:GASP}, we have then proved that the Mutex Watershed algorithm is part of a larger generalized framework, named GASP, for agglomerative clustering of graphs with both positive and negative edge weights. We have analyzed several theoretical and empirical properties of the algorithms in it, by exploring new and existing combinations of linkage criteria and applying them to different types of graphs. Algorithms based on an average linkage criterion proved to be simple and robust approaches when applied to an instance segmentation task. On large volume images, these simple average agglomeration algorithms achieved state-of-the-art results on the competitive CREMI neuron segmentation challenge. 

Finally, in Chapter \ref{chapter:LSI}, we introduced a novel proposal-free method predicting encoded single-instance masks in a sliding window style, one for each pixel of the input image, and introduced a parameter-free approach to aggregate predictions from overlapping masks and obtain all instances concurrently. The resulting pipeline proved to be strongly robust to noise when applied to large neuron segmentation volumetric images. 
The method also endows its predictions with an uncertainty measure, depending on the consensus of the overlapping  single-instance masks. Future work could use these uncertainty measures to estimate the confidence of individual instances, which may help to facilitate the subsequent proof-reading step still needed in neuron segmentation.

The proposed unifying framework for agglomerative clustering algorithms also opens several opportunities for further research. Future work could investigate which objectives are optimized by the algorithms included in the GASP framework, and study how they relate to the multicut / correlation clustering problem. A generalized GASP objective function could also be linked to existing cost functions for hierarchical clustering on unsigned graphs with positive edge weights \cite{moseley2017approximation,cohen2019hierarchical,dasgupta2016cost}. 
Another research direction is the structured learning of graph edge weights, which optimizes the segmentation performance directly. The fact that this approach has already been successfully applied to instance segmentation and other graph partitioning algorithms \cite{funke2018large,kong2018recurrentPix,wolf2017learned,cerrone2019end} suggests that it could be similarly used to achieve an end-to-end training of the segmentation pipelines presented in this thesis that rely on agglomerative algorithms in the GASP framework.
